% billiards: Program to model collisions between 2d discs
% Copyright (C) 2022  Tom Spencer (tspencerprog@gmail.com)
%
% This file is part of billiards
%
% billiards is free software: you can redistribute it and/or modify
% it under the terms of the GNU General Public License as published by
% the Free Software Foundation, either version 3 of the License, or
% (at your option) any later version.
%
% This program is distributed in the hope that it will be useful,
% but WITHOUT ANY WARRANTY; without even the implied warranty of
% MERCHANTABILITY or FITNESS FOR A PARTICULAR PURPOSE.  See the
% GNU General Public License for more details.
%
% You should have received a copy of the GNU General Public License
% along with this program.  If not, see <https://www.gnu.org/licenses/>.

\documentclass{article}
\usepackage[utf8]{inputenc}
\usepackage[a4paper, total={6in, 8in}]{geometry}
\usepackage{amsmath}
\usepackage{amssymb}
\usepackage{bm}
\usepackage{datetime}

\def\bibsection{\section*{References}} 

% differential operator
\newcommand{\diff}{\textnormal{d}}
\DeclareMathOperator{\sign}{sign}

\title{Billiards}
\author{Tom Spencer}

\begin{document}

\maketitle

\section{Introduction}
Billiards aims to model a two dimensional gas by directly modelling motions of individual particles and the collisions between them. We then hope to extract some interesting physics from such a simulation, such as showing speed distribution is a Maxwell-Boltzmann, computing the heat capacity and explore its equation of state.

There are two approaches to such a simulation. A time driven approach is the most simplest, where we advance the simulation forward by a small time increment $\diff t$ and then check for and perform any collisions that took place during $\diff t$. This approach is simple to understand and implement but has some drawbacks, namely:
\begin{itemize}
    \item If we choose too large a time step, fast moving particles can skip over and avoid collisions with other particles
    \item This means a small step must be selected which is then computationally expensive, especially considering most particles aren't colliding
\end{itemize}

However, unlike many N-body simulation, each particle only experiences a force during a collisions, meaning they travel in straight lines in between collisions. This motivates the event-driven approach, where we have a variable time step, jumping from one collision to the next. It can be summarised as:
\begin{itemize}
    \item Determine the next particle to experience a collision and the time $t_1$ it occurs, current time is $t_0$
    \item Advance all particles to the time $t_1$
    \item Perform the collision on the particle (e.g. flip perpendicular component of velocity)
    \item Store particle ID, time of collision, new position and velocity as an event
    \item Repeat
\end{itemize}

This approach has several advantages, we cannot have issues with fast moving particles skipping collisions and we don't have to perform many iterations where no collisions occurred where one larger jump is sufficient. It can also deal with simple forces acting on particles such that their trajectory between collisions is known beforehand, e.g. constant gravity. However it is more complex to implement and cannot deal with more complicated forces, e.g. Newtonian gravity. Since Billiards does not aim to model complicated forces, it uses the event-driven approach.

\section{Theory}

In the following, a particle starts at position $\vec{r}_0$ and travels with velocity $\vec{v}$ and in an external, uniform gravitational field. Its position is therefore given by,
\begin{equation}
    \vec{r}_p(t)
    =
    \vec{r}_0 + \vec{v}t + \frac{1}{2} \vec{g} t^2,
\end{equation}
where $\vec{g}$ is the acceleration due to gravity.
There are three types of collisions we must consider:
\begin{itemize}
    \item Disc-disc collisions
    \item Disc-wall collisions
    \item Disc-boundary collisions
\end{itemize}
We first derive how to test for such collisions and then derive the physics of the collisions.

\subsection{Testing for collisions}

\subsubsection{Testing for Disc-wall collisions}
A wall that starts at a point $\vec{A}$ and ends at $\vec{B}$ can be parameterised as,
\begin{equation}
    \vec{r}_w(s)
    =
    \vec{A} + (\vec{B} - \vec{A}) s,
\end{equation}
with $0 \leq s \leq 1$. There are two ways a particle can collide with a wall. It can either collide with the ends at $\vec{A}$ or $\vec{B}$, or alternatively it can collide with main body of the wall, corresponding to $0 < s < 1$ in the above parametrisation. Initially, we only consider particles in a rectangular box. As such, particles cannot collide with the ends and so we initially only test for the second scenario.

When a particle of radius $R$ collides with the wall, the perpendicular distance between the wall and the centre of the particle is precisely $R$. The collision takes place when,
\begin{equation}
\label{eq:wall_collision}
    (\vec{r}_p(t) - \vec{r}_w(s))^2 = R^2,
\end{equation}
and occurs at,
\begin{equation}
    s
    =
    \frac{(\vec{r}_p(t) - \vec{A}) \cdot (\vec{B} - \vec{A})}{|\vec{B} - \vec{A}|^2}.
\end{equation}
Substituting in for $s$, we can express $\vec{r}_p(t) - \vec{r}_w(s)$ at the moment of collision as,
\begin{align}
    \vec{r}_p(t) - \vec{r}_w(s)
    &=
    \left(
        \vec{r}_0 - \vec{A}
    \right)
    -
    \vec{\gamma}
    \left(
        \vec{r}_0 - \vec{A}
    \right) \cdot \vec{\gamma}
    +
    t 
    \left(
        \vec{v}
        -
        (\vec{v} \cdot \vec{\gamma}) \vec{\gamma}
    \right)
    +
    \frac{t^2}{2} 
    \left(
        \vec{g}
        -
        (\vec{g} \cdot \vec{\gamma}) \vec{\gamma}
    \right), \\
    &=
    \vec{n} \cdot
    \left(
        \vec{r}_0 - \vec{A}
    \right) \vec{n}
    +
    t 
    (\vec{v} \cdot \vec{n}) \vec{n}
    +
    \frac{t^2}{2} 
    (\vec{g} \cdot \vec{n}) \vec{n},
\end{align}
where $\vec{\gamma}$ is the unit vector along the wall from $\vec{A}$ to $\vec{B}$,
\begin{equation}
    \vec{\gamma} = \frac{\vec{B} - \vec{A}}{|\vec{B}-\vec{A}|},
\end{equation}
and $\vec{n}$ is a unit vector normal to the wall and orthogonal to $\vec{\gamma}$. Substituting back into Eq~\ref{eq:wall_collision} and simplifying produces two quadratics in $t$,
\begin{equation}
    \frac{\vec{g} \cdot \vec{n}}{2} t^2
    +
    (\vec{v} \cdot \vec{n}) t
    +
    (\vec{r}_0 - \vec{A}) \cdot \vec{n}
    \pm R
    =
    0.
\end{equation}

We first consider the case $\vec{g} \cdot \vec{n} = 0$. The solution for $t$ becomes,
\begin{equation}
    t
    =
    \frac{(\vec{A} - \vec{r}_0) \cdot \vec{n} \pm R}{\vec{v} \cdot \vec{n}}.
\end{equation}
If we choose $\vec{n}$ such that $(\vec{A} - \vec{r}_0) \cdot \vec{n} > 0$, then only the negative solution can possibly be physical as the disc collides with the wall on the same side the disc currently resides. We therefore find $t$ as,
\begin{equation}
\label{eq:wall_collision_sol_g_zero}
    t
    =
    \frac{(\vec{A} - \vec{r}_0) \cdot \vec{n} - R}{\vec{v} \cdot \vec{n}}.
\end{equation}
We must be careful if a disc-wall collision has just occurred. As the disc will just be touching the wall, $t$ will be close to 0. This is easily resolved since the component of $\vec{v}$ perpendicular to the wall should point away from the wall after the collision. We therefore reject any collision where $\vec{v} \cdot \vec{n} \leq 0$.

For the case $\vec{g} \cdot \vec{n} \neq 0$, applying the standard quadratic formula produces, 
\begin{equation}
    t
    =
    \frac{ - \vec{v} \cdot \vec{n} \pm \sqrt{ (\vec{v} \cdot \vec{n})^2 + 2(\vec{g} \cdot \vec{n})((\vec{A} - \vec{r}_0) \cdot \vec{n} \pm R)}}{\vec{g} \cdot \vec{n}}.
\end{equation}
Again, if we choose $\vec{n}$ such that $(\vec{A} - \vec{r}_0) \cdot \vec{n} > 0$ we need only consider the negative solution for the second $\pm$. By considering the sign of $\vec{g} \cdot \vec{n}$, we see for a collision in the future, we will almost always want the positive solution for the first sign,
\begin{equation}
\label{eq:wall_collision_sol_g_non_zero}
    t
    =
    \frac{ - \vec{v} \cdot \vec{n} + \sqrt{ (\vec{v} \cdot \vec{n})^2 + 2(\vec{g} \cdot \vec{n})((\vec{A} - \vec{r}_0) \cdot \vec{n} \pm R)}}{\vec{g} \cdot \vec{n}}.
\end{equation}
The exception is when we have just processed a disc-wall collision. If we are `above' the wall ($\vec{g} \cdot \vec{n} > 0$) then Eq~\ref{eq:wall_collision_sol_g_non_zero} will produce the correct, later time of next collision since after the collision $\vec{v} \cdot \vec{n} < 0$.

However, if we are `below' the wall ($\vec{g} \cdot \vec{n} < 0$) the disc will not collide with the wall again and Eq~\ref{eq:wall_collision_sol_g_non_zero} will give a spurious solution. We therefore set $t = \infty$ if both $\vec{v} \cdot \vec{n} < 0$ and $\vec{g} \cdot \vec{n} < 0$.

Finally we note we have assumed the disc can only collide with the wall on the same side it currently resides. This is true for $\vec{g} \cdot \vec{n} = 0$ or if both of the wall's end points are outside the bounds of the simulation. In future we plan to allow walls with end points within the bounds of the simulation. When we do, we will need to consider both sides.

\subsubsection{Testing for Disc-boundary collisions}
Boundaries are used to split the simulation area into rectangular sectors in order to improve efficiency. They are straight lines and are described in the same manner as walls. However, a disc-boundary collision occurs when the centre of a disc intersects with the boundary. We can find the collision time from Eq~\ref{eq:wall_collision_sol_g_zero} if $\vec{g} \cdot \vec{n} = 0$ and Eq~\ref{eq:wall_collision_sol_g_non_zero} otherwise by setting $R = 0$. For $\vec{g} \cdot \vec{n} = 0$ we get,
\begin{equation}
    t
    =
    \frac{(\vec{A} - \vec{r}_0) \cdot \vec{n}}{\vec{v} \cdot \vec{n}},
\end{equation}
and for $\vec{g} \cdot \vec{n} \neq 0$ we get,
\begin{equation}
\label{eq:disc_boundary_g_not_zero}
    t
    =
    \frac{ - \vec{v} \cdot \vec{n} + \sqrt{ (\vec{v} \cdot \vec{n})^2 + 2(\vec{g} \cdot \vec{n})(\vec{A} - \vec{r}_0) \cdot \vec{n}}}{\vec{g} \cdot \vec{n}}.
\end{equation}

We must again be careful if we have just processed a disc-boundary collision and are currently sitting on the boundary. Finite numerical precision means the centre of the disc my not be exactly on the boundary but to either side. This could leave us stuck in a loop where we endlessly process the same boundary collision. Previously we solved this by not testing a boundary if the disc's last event was a collision with said boundary. However gravity makes this a little more complicated as a disc may interact with the same boundary sequentially. That is, it may interact with the same boundary twice with no intervening disc-disc or disc-wall events.

We resolve this by noting if the disc-boundary collision has not been processed, the disc's velocity at the point of collision is such it is leaving the sector currently listed as its sector ID. By contrast, if the disc-boundary collision has been processed, it will be entering the sector listed as its sector ID. We do this by constructing a vector normal to the boundary and orientated such that it points `towards' the centre of the sector. A disc is leaving the sector it is in if this normal vector dotted with the disc's velocity at the time of collision is less than zero. 

Therefore for the case $\vec{g} \cdot \vec{n} = 0$, we only need to process a disc-boundary collision if the disc is leaving its current sector. For the case $\vec{g} \cdot \vec{n} \neq 0$, we compute the disc's velocity at the solution give by Eq~\ref{eq:disc_boundary_g_not_zero} and check if the disc is leaving its current sector. If it is, this time is the time to the next boundary. Otherwise we perform the same computation for the second solution. If this solution also `not leaving', we return the time to next collision as infinity.

We lastly note $t$ must be positive for a collision to occur and that since the boundary endpoints are placed outside the simulation bounds, there is no need to ensure $0 \leq s \leq 1$.

\subsubsection{Testing for disc-disc collisions}
Consider two discs $i, j$ with radii $R_i$, $R_j$ respectively. Denoting the position of each particle as $\vec{r}_i(t)$ and $\vec{r}_j(t)$, a collision occurs when,
\begin{equation}
    (\vec{r}_i(t) - \vec{r}_j(t))^2 = (R_i + R_j)^2.
\end{equation}
We can rewrite this as,
\begin{equation}
    (t\vec{\alpha} + \vec{\beta})^2 = R^2,
\end{equation}
with,
\begin{align}
    \vec{\alpha}
    =
    \vec{v}_i - \vec{v}_j,
    &&
    \vec{\beta}
    =
    \vec{r}_{i0} - \vec{r}_{j0},
    &&
    R = R_i + R_j.
\end{align}
Here $\vec{v}_i$, $\vec{v}_j$ and $\vec{r}_{i0}$, $\vec{r}_{j0}$ are the velocity and initial position of each respective disc. Applying the standard quadratic formula produces,
\begin{equation}
    t
    =
    \frac{1}{\alpha^2}
    \left(
        -\vec{\alpha} \cdot \vec{\beta}
        \pm
        \sqrt{(\vec{\alpha} \cdot \vec{\beta})^2 + \alpha^2(R^2 - \beta^2)}
    \right).
\end{equation}
We may assume $t \geq 0$ for both solutions, if there are real solutions, since discs may not overlap. Of two solutions, the negative corresponds to the true collision, the positive to the case of one disc passing through the other and coming out the other side. We are therefore only interested in the negative of the two solutions.

However, the above has issues when the discs are very close after performing a collision. We therefore use the approach described in \cite{numerical-recipes} to solve the quadratic $ax^2 + bx + c = 0$ as,
\begin{align}
    x_1 = \frac{q}{a}, 
    &&
    x_2 = \frac{c}{q},
    &&
    q = - \frac{1}{2}\left(
        b
        +
        \sign(b)
        \sqrt{b^2 - 4ac}
    \right),
\end{align}
where we have,
\begin{align}
    a
    =
    \alpha^2,
    &&
    b
    =
    2 \vec{\alpha} \cdot \vec{\beta},
    &&
    c
    =
    \beta^2 - R^2.
\end{align}
Noting $\vec{\alpha} \cdot \vec{\beta}$ must be less than zero for a collision to occur, we find $t$ as,
\begin{equation}
    t
    =
    \frac{c}{-b + \sqrt{b^2-4ac}}.
\end{equation}
This only holds so long as $\vec{\alpha} \cdot \vec{\beta} < 0$, if it is not this will not produce the correct time and we should return $t = \infty$.

\subsection{Processing collisions}
We now discuss the implemented physics of disc-disc and disc-wall collisions. Note that no processing is required for disc-boundary collisions, at least in terms of the physics. Since we model instantaneous collisions, we neglect the effects of gravity.

\subsubsection{Processing disc-disc collisions}
We model the collision using constant coefficients of normal restitution, $\epsilon_\mathrm{n}$, and tangential restitution, $\epsilon_\mathrm{t}$, as described in \cite{Kremer-2014}. This approach defines the behaviour of the collision in terms of the relative velocity of the point on each disk that touches the other during the collision, $\vec{g}_{ij}$. The resulting states of discs $i$ and $j$ are then determined by,
\begin{align}
\label{eq:relative-velocity-relations}
    \left( \vec{g}'_{ij} \cdot \hat{\vec{\sigma}}_{ij} \right) \hat{\vec{\sigma}}_{ij}
    =
    - \epsilon_\mathrm{n} \left( \vec{g}_{ij} \cdot \hat{\vec{\sigma}}_{ij} \right) \hat{\vec{\sigma}}_{ij},
    &&
    \left( \vec{g}'_{ij} \times \hat{\vec{\sigma}}_{ij} \right) \times \hat{\vec{\sigma}}_{ij}
    =
    - \epsilon_\mathrm{t} \left( \vec{g}'_{ij} \times \hat{\vec{\sigma}}_{ij} \right) \times \hat{\vec{\sigma}}_{ij},
\end{align}
where $\hat{\vec{\sigma}}_{ij}$ is a unit vector pointing from the centre of disk $i$ to the centre of disk $j$ at the moment of collision,
\begin{equation}
    \hat{\vec{\sigma}}_{ij} = \frac{\vec{r}_j - \vec{r}_i}{|\vec{r}_j - \vec{r}_i|}.
\end{equation}
We note the sign convention in Eq~\ref{eq:relative-velocity-relations} varies but the use here is consistent with \cite{Kremer-2014}.

We now broadly reproduce the derivation of \cite{Kremer-2014} for the velocities and angular velocities after the collision in a little more detail and extend it to particles with differing masses, moments of inertia and radii. We denote the moment of inertia of each disc $I_i$, $I_j$ and angular velocity $\vec{\omega}_i$, $\vec{\omega}_j$. We will again work in the centre of mass frame where the total linear momentum is zero.

If particles $i$ and $j$ are colliding, and as a result of the collision particle $i$ experiences an impulse $\vec{Q}$, the new momenta of the particles are trivially, 
\begin{align}
\label{eq:momenta-relations}
    \vec{P}'_i = \vec{P}_i + \vec{Q},
    &&
    \vec{P}'_j = \vec{P}_j - \vec{Q}.
\end{align}

For the new angular velocities, we note particle $i$ experiences a force acting on the surface of the point of collision. This force provides disc $i$ with the impulse $\vec{Q}$, but it also produces a torque and therefore an angular impulse,
\begin{equation}
    R_i \hat{\vec{\sigma}}_{ij} \times \vec{Q},
\end{equation}
This torque acts solely to change $\vec{\omega}_i$. Hence we may express the new angular velocities as,
\begin{align}
\label{eq:ang-velocity-relations}
    \vec{\omega}'_i = \vec{\omega}_i + \frac{R_i}{I_i} \hat{\vec{\sigma}}_{ij} \times \vec{Q},
    &&
    \vec{\omega}'_j = \vec{\omega}_j + \frac{R_j}{I_j} \hat{\vec{\sigma}}_{ij} \times \vec{Q},
\end{align}
where we have used $\hat{\vec{\sigma}}_{ji} = -\hat{\vec{\sigma}}_{ij}$ for $\vec{\omega}'_j$.

As a sanity check, we may check momentum and angular momentum are indeed conserved. Momentum is clearly conserved. The angular momentum is then,
\begin{align}
    \vec{r}_i \times \vec{P}_i + \vec{r}_j \times \vec{P}_j +
    I_i \vec{\omega}_i + I_j \vec{\omega}_j =& \,
    \vec{r}_i \times \left( \vec{P}_i + \vec{Q} \right) + 
    \vec{r}_j \times \left( \vec{P}_j - \vec{Q} \right) + \\
    & I_i \left( \vec{\omega}_i + \frac{R_i}{I_i} \hat{\vec{\sigma}}_{ij} \times \vec{Q} \right) +
    I_j \left( \vec{\omega}_j + \frac{R_j}{I_j} \hat{\vec{\sigma}}_{ij} \times \vec{Q} \right).
\end{align}
Simplifying yields,
\begin{equation}
    \left( \vec{r}_j - \vec{r}_i \right) \times \vec{Q}
    =
    \left( R_i + R_j \right) \hat{\vec{\sigma}}_{ij} \times \vec{Q},
\end{equation}
which is identical by the definition of $\hat{\vec{\sigma}}_{ij}$.

The relative velocity at the point of collision can be written as,
\begin{align}
    \vec{g}_{ij}
    &=
    \vec{v}_i - \vec{v}_j + 
    \left(
        R_i \vec{\omega}_i + R_j \vec{\omega}_j
    \right) \times \hat{\vec{\sigma}}_{ij}, \\
    &=
    \frac{m_i + m_j}{m_i m_j}
    \vec{P}_i
    + 
    \left(
        R_i \vec{\omega}_i + R_j \vec{\omega}_j
    \right) \times \hat{\vec{\sigma}}_{ij},
\end{align}
where we have used the fact we are working in the centre of mass frame to write $\vec{P}_j = - \vec{P}_i$. Using the expressions for the final momenta and angular velocities in Eq~\ref{eq:momenta-relations} and Eq~\ref{eq:ang-velocity-relations}, we can write the relative velocity at the point of collision after the collision as,
\begin{align}
    \vec{g}'_{ij}
    &=
    \frac{m_i + m_j}{m_i m_j}
    \vec{P}'_i
    + 
    \left(
        R_i \vec{\omega}'_i + R_j \vec{\omega}'_j
    \right) \times \hat{\vec{\sigma}}_{ij}, \\
    &=
    \frac{m_i + m_j}{m_i m_j}
    \left(
        \vec{P}_i + \vec{Q}
    \right)
    + 
    \left(
        R_i 
        \left[
            \vec{\omega}_i + \frac{R_i}{I_i} \hat{\vec{\sigma}}_{ij} \times \vec{Q}
        \right]
        +
        R_j
        \left[
            \vec{\omega}_j + \frac{R_j}{I_j} \hat{\vec{\sigma}}_{ij} \times \vec{Q}
        \right]
    \right) \times \hat{\vec{\sigma}}_{ij}, \\
    &=
    \vec{g}_{ij}
    +
    \frac{m_i + m_j}{m_i m_j} \vec{Q}
    +
    \left(
        \frac{R_i^2}{I_i} + \frac{R_j^2}{I_j}
    \right)
    \left(
        \hat{\vec{\sigma}}_{ij} \times \vec{Q}
    \right)
    \times \hat{\vec{\sigma}}_{ij}
\end{align}

We now seek to find an expression for $\vec{Q}$. Substituting $\vec{g}'_{ij}$ into the first relation of Eq~\ref{eq:relative-velocity-relations} gives,
\begin{equation}
\label{eq:Q-dot-sigma}
    \vec{Q} \cdot \hat{\vec{\sigma}}_{ij}
    =
    -(1 + \epsilon_\mathrm{n})
    \frac{m_i m_j}{m_i + m_j}
    \vec{g}_{ij} \cdot \hat{\vec{\sigma}}_{ij}.
\end{equation}
From the second relation of Eq~\ref{eq:relative-velocity-relations} and using that for some vector $\vec{A}$ we have $(\vec{A} \times \hat{\vec{\sigma}}_{ij}) \times \hat{\vec{\sigma}}_{ij} = (\vec{A} \cdot \hat{\vec{\sigma}}_{ij}) \hat{\vec{\sigma}}_{ij} - \vec{A}$, we have,
\begin{align}
    - \epsilon_\mathrm{t} (\vec{g}_{ij} \times \hat{\vec{\sigma}}_{ij}) \times \hat{\vec{\sigma}}_{ij}
    =& \,
    (\vec{g}_{ij} \times \hat{\vec{\sigma}}_{ij}) \times \hat{\vec{\sigma}}_{ij}
    +
    \frac{m_i + m_j}{m_i m_j}
    (\vec{Q} \times \hat{\vec{\sigma}}_{ij}) \times \hat{\vec{\sigma}}_{ij} \\
     &+ 
    \left(
        \frac{R_i^2}{I_i} + \frac{R_j^2}{I_j}
    \right)
    (\vec{Q} \times \hat{\vec{\sigma}}_{ij}) \times \hat{\vec{\sigma}}_{ij},
\end{align}
which leads to,
\begin{equation}
    \vec{Q} - (\vec{Q} \cdot \hat{\vec{\sigma}}_{ij}) \hat{\vec{\sigma}}_{ij}
    =
    -(1+\epsilon_\mathrm{t})
    \left[
        \frac{m_i + m_j}{m_i m_j}
        +
        \frac{R_i^2}{I_i} + \frac{R_j^2}{I_j}
    \right]^{-1}
    \left(
        \vec{g}_{ij} - (\vec{g}_{ij} \cdot \hat{\vec{\sigma}}_{ij}) \hat{\vec{\sigma}}_{ij}
    \right)
\end{equation}
Combining this with Eq~\ref{eq:Q-dot-sigma} produces an expression for $\vec{Q}$,
\begin{equation}
\label{eq:disc-disc-Q}
    \vec{Q}
    =
    -(1 + \epsilon_\mathrm{n})
    \frac{m_i m_j}{m_i + m_j}
    (\vec{g}_{ij} \cdot \hat{\vec{\sigma}}_{ij}) \hat{\vec{\sigma}}_{ij}
    -
    (1+\epsilon_\mathrm{t})
    \left[
        \frac{m_i + m_j}{m_i m_j}
        +
        \frac{R_i^2}{I_i} + \frac{R_j^2}{I_j}
    \right]^{-1}
    \left(
        \vec{g}_{ij} - (\vec{g}_{ij} \cdot \hat{\vec{\sigma}}_{ij}) \hat{\vec{\sigma}}_{ij}
    \right).
\end{equation}
Although we derived this expression for $\vec{Q}$ in the centre of mass frame, since $\vec{g}_{ij}$ and $\hat{\vec{\sigma}}_{ij}$ only depend on the relative velocity and position, it will be the same in all inertial frames. Similarly, the expressions in Eq~\ref{eq:momenta-relations} and Eq~\ref{eq:ang-velocity-relations} will also be the same in all frames. We therefore do not need to transform into the centre of mass frame and back to use it.

The ranges of $\epsilon_\mathrm{n}$ and $\epsilon_\mathrm{t}$ are $0 \leq \epsilon_\mathrm{n} \leq 1$ and $-1 \leq \epsilon_\mathrm{t} \leq 1$. $\epsilon_\mathrm{t} = -1$ corresponds to so called perfectly smooth discs whereas $\epsilon_\mathrm{t} = 1$ to so called perfectly rough. In the case where $\epsilon_\mathrm{n} = 1$ and $\epsilon_\mathrm{t} = \pm 1$ it can be shown energy is conserved during the collision. $\vec{Q}$ reduces to an expression consistent with perfectly smooth discs in the case $\epsilon_\mathrm{n} = 1$, $\epsilon_\mathrm{t} = -1$ as expected. It also reduces to an expression consistent with Eq~2.7 of \cite{Kremer-2014} in the case of disks with identical masses, radii and moments of inertia.

\subsubsection{Processing disc-wall collisions}
A disc-wall collision can be modelled as a disc-disc collision where the disc that represents the wall has infinite mass/moment of inertia and zero velocity/angular velocity. In this limit, Eq~\ref{eq:disc-disc-Q} becomes (dropping subscript $i, j$),
\begin{equation}
    \vec{Q}
    =
    -(1 + \epsilon_\mathrm{n})
    m
    (\vec{g} \cdot \hat{\vec{\sigma}}) \hat{\vec{\sigma}}
    -
    (1+\epsilon_\mathrm{t})
    \left[
        \frac{1}{m}
        +
        \frac{R^2}{I}
    \right]^{-1}
    \left(
        \vec{g} - (\vec{g} \cdot \hat{\vec{\sigma}}) \hat{\vec{\sigma}}
    \right),
\end{equation}
where $\vec{g} = \vec{v} + R \vec{\omega} \times \hat{\vec{\sigma}}$ and $\hat{\vec{\sigma}}$ is a unit vector pointing from the centre of the disc to the point on it's surface colliding with the wall.

\section{Insights from statistical mechanics}
Here we discuss some of the macroscopic physics of a system of billiards. We leave a discussion of the effects of gravity for later.

The partition function for a N-body system can be written as,
\begin{equation}
    Z_N
    =
    \frac{1}{h^{3N} N!}
    \int \diff^2 \vec{q}_1 ... \diff^2 \vec{q}_N \, \diff^2 \vec{p}_1 ... \diff^2 \vec{p}_N \,
    \exp{\left(
        - \beta \, \mathcal{H}(\vec{q}_1,..., \vec{q}_N, \vec{p}_1, ..., \vec{p}_N)
    \right)},
\end{equation}
where $\beta = \frac{1}{k_B T}$. The Hamiltonian can be written as,
\begin{equation}
    \mathcal{H}(\vec{q}_1,..., \vec{q}_N, \vec{p}_1, ..., \vec{p}_N)
    =
    \sum_{j=1}^N \frac{p_j^2}{2m}
    +
    U(\vec{q}_1,..., \vec{q}_N),
\end{equation}
where $U(\vec{q}_1,..., \vec{q}_N)$ si,
\begin{equation}
    U(\vec{q}_1,..., \vec{q}_N)
    =
    \begin{cases}
       \infty & \text{if any } |\vec{q}_i - \vec{q}_j|< 2 R \\
       \infty & \text{if any disc is penetrating a wall} \\
       0 & \text{otherwise}
    \end{cases}.
\end{equation}
The partition function can then be rewritten as,
\begin{align}
\label{eq:partition_function_simplified}
    Z_N
    &=
    \frac{V^N Q}{h^{3N} N!}
    \left(
        \int \diff^2 \vec{p} \, \exp{
        \left(- \frac{\beta p^2}{2m} \right)}
    \right)^N, \\
    &=
    \frac{V^N Q}{h^{3N} N!}
    \left(
        \frac{2 \pi m}{\beta}
    \right)^N
\end{align}
where $Q$ is the configuration integral,
\begin{equation}
\label{eq:configuration_integral}
    \frac{1}{V^N}\int \diff^2 \vec{q}_1 ... \diff^2 \vec{q}_N \,
    \exp{\left( -\beta \, U(\vec{q}_1, ..., \vec{q}_N) \right)},
\end{equation}
which is taken over the container the gas is in. 

\subsection{Equation of state}
Evaluating $Q$ is highly non-trivial, so we approximate the equation of state at low densities using the virial expansion,
\begin{equation}
    \frac{pV}{N k_B T}
    =
    1
    +
    \sum_{k=2}^\infty B_k(T) \left( \frac{N}{V} \right)^{k-1}.
\end{equation}
In two dimensions, the first virial coefficients for hard disks of area $A$ are \cite{virial-coeff},
\begin{align}
    B_2(T) = 2A,
    &&
    B_3(T) = \left( \frac{16}{3} - \frac{4 \sqrt{3}}{\pi} \right) A^2,
    &&
    B_4(T) = \left( 16 - \frac{36 \sqrt{3}}{\pi} + \frac{80}{\pi^2} \right) A^3.
\end{align}

\subsection{Speed distribution}
The probability of the system occupying a given state $(\vec{q}_1, .., \vec{q}_N, \vec{p}_1, ..., \vec{p}_N)$ is,
\begin{equation}
    P(\vec{q}_1, ..., \vec{q}_N, \vec{p}_1, ..., \vec{p}_N) \, \diff \Phi
    =
    \frac{\exp{\left(
        - \beta \, \mathcal{H}(\vec{q}_1,..., \vec{q}_N, \vec{p}_1, ..., \vec{p}_N)
    \right)}}{Z_N} \,
    \diff \Phi,
\end{equation}
where $\diff \Phi =\diff^2 \vec{q}_1 ... \diff^2 \, \vec{q}_N  \diff^2 \vec{p}_1 ... \diff^2 \vec{p}_N$. We can determine the momentum distribution by integrating over all $\vec{q}_1,..., \vec{q}_N$ and $\vec{p}_2, ..., \vec{p}_N$ and then integrating out the angular dependence using $\diff^2 \vec{p} = \diff p \diff \theta p$. Dropping the subscript 1 produces,
\begin{align}
    P(p) \, \diff p
    &=
    \left(
        \frac{\beta}{2 \pi m}
    \right) \diff p
    \int \diff \theta \,
    p
    \exp{\left( -\frac{\beta p^2}{2m} \right)}, \\
    &=
    \frac{\beta p}{m}
    \exp{\left( -\frac{\beta p^2}{2m} \right)} \diff p.
\end{align}
The expected speed distribution is then,
\begin{equation}
    P(v) \, \diff v
    =
    \beta m v \exp{\left( -\frac{\beta p^2}{2m} \right)} \diff v.
\end{equation}

\section{Program structure}
Here we describe a rough outline of the structure of Billiards. The simulation itself is written in C++ and a Cython wrapper allows the simulation code to be setup, called \& analysed from Python. On the C++ side, we have:

\begin{itemize}
    \item A \texttt{Vec2D} class to represent a 2 dimensional real vector.
    \begin{itemize}
        \item Includes methods for magnitude, magnitude squared and dot product
    \end{itemize}
    \item A \texttt{Disc} class to represent a disc
    \begin{itemize}
        \item Stores position, velocity, mass, radius etc.
    \end{itemize}
    \item A \texttt{Wall} class to represent a straight wall a particle can bounce off
    \begin{itemize}
        \item Stores start and end points, $\vec{A}$ and $\vec{B}$ of the wall
        \item Stores a unit vector pointing from the start point to the wall's end point 
    \end{itemize}
    \item An \texttt{Event} class to represent either a disc-wall or disc-disc collision
    \begin{itemize}
        \item Stores index of the disc in the simulation's vector of discs, along with time of event, current position, new velocity/angular velocity, partner index etc.
    \end{itemize}
    \item A \texttt{Sim} class to represent a simulation
    \begin{itemize}
        \item Vector to store the discs that are in the simulation, both in their current state and their initial state
        \item Vector to store the walls in the simulation
        \item Vector to store the events that occur during the simulation, if desired
        \item Current time of the simulation
        \item A method \texttt{advance()} to advance the simulation by either a given number of events or duration, whichever is reached sooner, should have an argument to set whether the events should be stored or only the current state should be stored
        \item A method \texttt{Add\_disc()} to add a disc to the simulation
        \item Getter/setter methods to get/set coefficients of normal/tangential restitution and the gravitational vector
    \end{itemize}
\end{itemize}

\subsection{Cython wrapper}
Here we summarise the structure of the Cython wrapper and interface to Python. For full details, see the documentation in \texttt{billiards.pyx}.
\begin{itemize}
    \item \texttt{PyEvent} wraps \texttt{Event}. It does not own the memory, that is owned by the C++ \texttt{Sim}.
    \begin{itemize}
        \item Exposes the member variables of an \texttt{Event} such as time, position etc. as properties
        \item New position and velocity are exposed as \texttt{numpy} arrays with shape \texttt{(2,)}
        \item Factory static method for creation from a \texttt{PySim} instance and index of event
    \end{itemize}
    \item \texttt{PySim} wrapper for \texttt{Sim}.
    \begin{itemize}
        \item Method \texttt{advance()} to advance the simulation by either a given number of collisions or duration, whichever is reached first.
        \item Methods \texttt{add\_disc()} and \texttt{add\_wall()} for adding discs/walls to the simulation individually
        \item Method \texttt{add\_random\_discs()} for adding discs in bulk randomly in a box, all with the given parameters e.g. mass, radius etc.
        \item Method \texttt{add\_box\_walls()} for adding four walls in the shape of a box, defined by the position of the bottom-left and top-right corners. Should not currently be used as disc collisions wit wall end points are not modelled.
        \item Property \texttt{initial\_state} to return the initial state of the system as a "state dictionary." Returns a dictionary where the keys correspond to names of disc properties (position, velocity, mass, etc.) and values are those properties as \texttt{numpy} arrays.
        \item Property \texttt{current\_state} that behaves the same as \texttt{initial\_state} but returns a state dictionary for the current state of the simulation.
        \item Properties for coefficients of normal/tangential restitution and $\vec{g}$
        \item Two generators for replaying the simulation
        \begin{itemize}
            \item These start at the beginning of the simulation and yield the same state dictionary of \texttt{numpy} arrays as \texttt{initial\_state} and \texttt{current\_state}
            \item The arrays in this dictionary are successively updated by reference
            \item \texttt{replay\_by\_event()} for moving forward from one event to the next
            \item \texttt{replay\_by\_time()} for moving forward at constant time intervals, useful for making an animation
        \end{itemize}
    \end{itemize}
\end{itemize}

\subsection{Algorithm}
We implement the approach described in \cite{Lubachevsky-1991}. Roughly, for each particle the algorithm keeps track of two events, The last collision the particle experienced and its next scheduled collision. The algorithm iterates by:
\begin{itemize}
    \item Determining the particle with the next scheduled event
    \item Making the scheduled event the most recent event the particle underwent 
    \item Determining the next collision of the particle and scheduling it
    \item If the next collision is disc-disc, appropriately scheduling the partner's new collision
\end{itemize}
The main computational effort is spent in determining the next collision and the next scheduled event. Determining the next time can be found in $\mathcal{O}(\log N)$ using a heap as suggested in \cite{Lubachevsky-1991}, whereas determining the next collision can be improved by partitioning space into sectors.

Billiards implements rectangular sectors, each of which should be large enough to fully contain any disc in the simulation. Ideally sectors should be sized so roughly one discs is in each sector. Only the sector the disc is currently in and those adjacent to it (8 for rectangles) need to be checked to determine the next possible disc-disc collision. Similarly, only sectors adjacent to a wall need to be checked for disc-wall interactions. There is some overhead as collision with sector boundaries must be checked and the current sector of each disc tracked. However, for large numbers of discs the reduced number of disc-disc or disc-wall collisions that need to be checked far outweighs this overhead.

% Bibliography
\bibliographystyle{ieeetr}
\bibliography{bibo}

\end{document}















